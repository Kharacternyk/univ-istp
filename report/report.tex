\documentclass[a4paper, 14pt]{extarticle}

\usepackage[ukrainian]{babel}

\usepackage{fontspec}
\setmainfont{Liberation Serif}

\usepackage[left=1.18in, right=0.59in, top=0.79in, bottom=0.79in]{geometry}

\usepackage{setspace}

\usepackage{tocloft}

\cftsetindents{section}{0em}{2em}
\cftsetindents{subsection}{0em}{2em}

\renewcommand\cfttoctitlefont{\hfill\Large\bfseries}
\renewcommand\cftaftertoctitle{\hfill\mbox{}}

\usepackage{etoolbox}
\patchcmd{\tableofcontents}{\contentsname}{\MakeUppercase\contentsname}{}{}

\let\Huge\normalsize
\let\huge\normalsize
\let\LARGE\normalsize
\let\Large\normalsize
\let\large\normalsize

\begin{document}
  \setstretch{1.5}

  \begin{titlepage}
    \begin{center}
      \textbf{КИЇВСЬКИЙ НАЦІОНАЛЬНИЙ УНІВЕРСИТЕТ} \\
      \textbf{ІМЕНІ ТАРАСА ШЕВЧЕНКА} \\
      Факультет комп'ютерних наук та кібернетики \vspace{5em}

      \textbf{Звіт} \\
      за спеціальністю 122 Комп'ютерні науки \\
      на тему: \\
      \textbf{Розробка вебзастосунку для клубу настільних ігор}
    \end{center}
    \vfill
    \begin{flushleft}
      Виконав студент 2-го курсу \\
      Віннічук Назар Дмитрович \vspace{2em}

      Науковий керівник: \\
      доцент, кандидат фіз.-мат. наук \\
      Омельчук Людмила Леонідівна
    \end{flushleft}
    \begin{flushright}
      Засвідчую, що у звіті немає запозичень з \\
      праць інших авторів без відповідних посилань.

      Студент \hspace{7.25em} Віннічук Н. Д.
    \end{flushright}
    \vspace{5em}
    \begin{center}
      Київ -- 2022
    \end{center}
  \end{titlepage}

  \begin{center}
    \textbf{РЕФЕРАТ}
  \end{center}

  НАСТІЛЬНІ ІГРИ, DJANGO, PYTHON, NIX, POSTGRESQL,
  BOOTSTRAP, ВЕБЗАСТОСУНОК, MVC, MODEL, VIEW, CONTROLLER, ORM.

  Об'єктом роботи є ознайомлення із сучасними вебтехнологіями на базі вебфреймворку
  Django та системою керування базами даних PostgreSQL.

  Предметом роботи є вебзастосунок з реляційною базою даних для забезпечення
  функціонування клубу настільних ігор.

  Метою роботи є створення функціонуючого вебзастосунку для адміністраторів та
  відвідувачів клубу настільних ігор.

  Інструменти розроблення: об'єктно-орієнтована мова програмування Python,
  вебфреймворк Django, система керування базами даних PostgreSQL, мова опису програмних
  пакетів та середовищ розробки Nix, мова середовища вебпереглядача JavaScript, мова
  опису структури вебсторінок HTML, мова опису стилів вебсторінок CSS, бібліотека
  готових стилів вебсторінок Bootstrap.

  Результати роботи: розроблено вебзастосунок на основі сучасних вебтехнологій,
  виконано роботу з проектування реляційної бази даних та візуального оформлення
  вебсторінок.

  Програмний продукт здатен використовуватись для реалізації функціонування
  клубу настільних ігор. Вимоги до користувачів: знання української мови (мови
  інтерфейсу застосунку).

  \pagebreak

  \tableofcontents
  \newpage

  \section{СКОРОЧЕННЯ ТА УМОВНІ ПОЗНАЧЕННЯ}
  \section{ВСТУП}
  \section{РОЗДІЛ 2. ОГЛЯД ВИКОРИСТАНИХ ТЕХНОЛОГІЙ}
  \subsection{Django}
  \subsection{PostgreSQL}
  \subsection{Nix}

\end{document}
