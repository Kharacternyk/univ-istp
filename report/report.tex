\documentclass[a4paper, 14pt]{extarticle}

\usepackage[ukrainian]{babel}

\usepackage{fontspec}
\setmainfont{Liberation Serif}

\usepackage[left=1.18in, right=0.59in, top=0.79in, bottom=0.79in]{geometry}

\usepackage{enumitem}
\usepackage{setspace}
\usepackage{indentfirst}
\usepackage{tocloft}

\usepackage{titlesec}
\titleformat{\section}[block]{\bfseries\filcenter}{}{1em}{}
\titleformat{\subsection}[block]{\bfseries\filcenter}{}{1em}{}

\renewcommand\cfttoctitlefont{\hfill\Large\bfseries}
\renewcommand\cftaftertoctitle{\hfill\mbox{}}

\addto\captionsukrainian{
  \renewcommand{\contentsname}{ЗМІСТ}
}

\renewcommand\cftsecdotsep{\cftdot}
\renewcommand\cftsubsecdotsep{\cftdot}
\renewcommand\cftsubsecfont{\normalfont\bfseries}
\renewcommand\cftsubsecpagefont{\normalfont\bfseries}
\renewcommand\cftsubsecleader{\bfseries\cftdotfill{\cftsecdotsep}}

\pagestyle{myheadings}
\setcounter{secnumdepth}{0}

\let\Huge\normalsize
\let\huge\normalsize
\let\LARGE\normalsize
\let\Large\normalsize
\let\large\normalsize

\begin{document}
  \setstretch{1.5}

  \begin{titlepage}
    \begin{center}
      \textbf{КИЇВСЬКИЙ НАЦІОНАЛЬНИЙ УНІВЕРСИТЕТ} \\
      \textbf{ІМЕНІ ТАРАСА ШЕВЧЕНКА} \\
      Факультет комп'ютерних наук та кібернетики \vspace{5em}

      \textbf{Звіт} \\
      за спеціальністю 122 Комп'ютерні науки \\
      на тему: \\
      \textbf{Розробка вебзастосунку для клубу настільних ігор}
    \end{center}
    \vfill
    \begin{flushleft}
      Виконав студент 2-го курсу \\
      Віннічук Назар Дмитрович \vspace{2em}

      Науковий керівник: \\
      доцент, кандидат фіз.-мат. наук \\
      Омельчук Людмила Леонідівна
    \end{flushleft}
    \begin{flushright}
      Засвідчую, що у звіті немає запозичень з \\
      праць інших авторів без відповідних посилань.

      Студент \hspace{7.25em} Віннічук Н. Д.
    \end{flushright}
    \vspace{5em}
    \begin{center}
      Київ -- 2022
    \end{center}
  \end{titlepage}

  \setcounter{page}{2}

  \begin{center}
    \textbf{РЕФЕРАТ}
  \end{center}

  НАСТІЛЬНІ ІГРИ, DJANGO, PYTHON, NIX, POSTGRESQL,
  BOOTSTRAP, ВЕБЗАСТОСУНОК, MVC, ORM.

  Об'єктом роботи є ознайомлення із сучасними вебтехнологіями на базі вебфреймворку
  Django та системою керування базами даних PostgreSQL.

  Предметом роботи є вебзастосунок з реляційною базою даних для забезпечення
  функціонування клубу настільних ігор.

  Метою роботи є створення функціонуючого вебзастосунку для адміністраторів та
  відвідувачів клубу настільних ігор.

  Інструменти розроблення: об'єктно-орієнтована мова програмування Python,
  вебфреймворк Django, система керування базами даних PostgreSQL, мова опису програмних
  пакетів та середовищ розробки Nix, пакетний менеджер Nix,
  мова середовища вебпереглядача JavaScript, мова
  опису структури вебсторінок HTML, мова опису стилів вебсторінок CSS, бібліотека
  готових стилів вебсторінок Bootstrap.

  Результати роботи: розроблено вебзастосунок на основі сучасних вебтехнологій,
  виконано роботу з проектування реляційної бази даних та візуального оформлення
  вебсторінок.

  Програмний продукт здатен використовуватись для реалізації функціонування
  клубу настільних ігор. Вимоги до користувачів: знання української мови (мови
  інтерфейсу застосунку).

  \clearpage

  \setstretch{1}
  \tableofcontents\thispagestyle{myheadings}
  \setstretch{1.5}

  \clearpage

  \section{СКОРОЧЕННЯ ТА УМОВНІ ПОЗНАЧЕННЯ}

  \section{ВСТУП}

  \textbf{Оцінка сучасного стану об'єкта розробки.}
  Ринок програмного забезпечення є наразі наповненим універсальними рішеннями,
  що намагаються покрити велику кількість сфер застосування одним програмним
  продуктом. Проте досі актуальною є потреба в вузькоспеціалізованому програмному
  забезпечені, що виконує лише одну функцію чи покриває лише одну сферу застосування,
  але робить це добре. Таким чином, альтернативні рішення можуть бути універсальнішими
  за наявний предмет розробки, але вони не передбачають вирішення
  для деяких специфічних проблем предметної області.

  \textbf{Актуальність роботи та підстави її виконання.}
  Актуальність роботи полягає в відсутності на ринку програмних продуктів рішень
  для ведення обліку клубів настільних ігор. Такі клуби вже присутні в Україні,
  тому потреба на такі програмні продукти викликана попитом ринку. Українська мова
  інтерфейсу є іще однією рисою програмного рішення, бажаною для українського ринку.


  \textbf{Мета й завдання роботи.}
  Метою роботи є розробка програмного продукту для застосування адміністраторами
  та гравцями клубу настільних ігор, а також ознайомлення з технологіями, що
  використовуються для розроблення вебзастосунків. Для досягнення мети було поставлено
  такі завдання:

  \begin{enumerate}[nosep, label=\arabic*)]
    \item описати середовище розробки за допомогою мови Nix;
    \item ознайомитися із системою керування базами даних PostgreSQL;
    \item спроєктувати базу даних для предметної області;
    \item створити проєкт вебфреймворку Django та приєднати до нього наявну базу даних;
    \item розробити валідацію моделей предметної області, логіку застосунку та зовнішній
      вигляд представлень.
  \end{enumerate}

  \textbf{Об’єкт, методи й засоби розроблення.}
  Об'єктом розроблення програмного продукту є вебзастосунок, що складається
  з адміністративного інтерфейсу для працівників клубу та зовнішнього інтерфейсу
  для гравців клубу. Адміністративний інтерфейс реалізує додавання, редагування,
  та видалення сутностей предметної області із бази даних застосунку. Серед
  сутностей предметної області є настільні ігри, їх локалізації, автори, ігрові сеанси,
  гравці клубу тощо. Зовнішній інтерфейс для гравців клубу реалізує перегляд наявних
  у клубі настільних ігор, перегляд супутньої до ігор інформації, а також створення
  ігрових сеансів.

  Серед інструментів, використаних для розроблення застосунку, є такі: текстовий редактор
  Vim із розширенням Python Language Server Protocol,
  пакетний менеджер Nix, вбудований командний рядок системи керування базами даних
  PostgreSQL.

  \textbf{Можливі сфери застосування.}
  Програмний продукт може бути використаним клубом настільних ігор, гравці та персонал
  якого володіють українською мовою.

  \clearpage

  \section{РОЗДІЛ 1. ОГЛЯД НАЯВНИХ НА РИНКУ РІШЕНЬ}
  \textbf{Strapi.} Система керування контентом з відкритим кодом.
  Користувач має описати сутності предметної області за допомогою мови JavaScript, після
  чого даний програмний продукт генерує інтерфейс адміністратора для додавання, видалення
  та редагування сутностей. Вбудована підтримка предметної області настільних ігор
  відсутня. Доступ до вебсторінки для перегляду сутностей без можливості видалення та
  редагування потребує створення облікового запису у системі. Можливості зміни вигляду
  вебсторінок обмежені.

  \textbf{Joomla.} Система керування контентом з відкритим кодом.
  Користувач має створити сутності предметної області за допомогою графічного
  інтерфейсу адміністративної панелі, після
  чого даний програмний продукт генерує інтерфейс адміністратора для додавання, видалення
  та редагування сутностей. Наявний вбудований конструктор гостьових сторінок.
  Вбудована підтримка предметної області настільних ігор відсутня. Зовнішній вигляд
  вебсторінок та логіка обробки даних може в певних межах бути змінена за допомогою
  графічного інтерфейсу, проте система додатків дозволяє подальшу зміну.


  \section{РОЗДІЛ 2. ОГЛЯД ВИКОРИСТАНИХ ТЕХНОЛОГІЙ}

  \subsection{2.1 Вебфреймворк Django}
  Django -- вебфреймворк високого рівня, що спонукає до швидкої розробки та чистого,
  прагматичного дизайну застосунку. Розробка вебфреймворку здійснюється волонтерами,
  а проєкт має відкритий вихідний код. Розробка вебзастосунків на основі
  вебфреймворку Django здійснюється об'єктно-орієнтованою мовою високого рівня Python.

  Django характеризується наявністю інтегрованих рішень для пересічних завдань
  веброзробки. Серед таких рішень є система автентифікації, ORM, рушій шаблонів HTML,
  та інтерфейс адміністратора, що здатен надати базовий функціонал керуванням
  сутностями предметної області за визначенням ORM моделей.

  \subsection{PostgreSQL}

  \subsection{Nix}

\end{document}
